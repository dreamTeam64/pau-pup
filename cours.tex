\documentclass[a4paper]{article}
\usepackage[utf8]{inputenc}
\usepackage[french]{babel}
\auhtor{Fontaine Pierre}
\title{Conception de programme}
\begin{document}
	\maketitle
	\newpage
	\tableofcontents
	\newpage
	\section{language graph vers Code C}
		Codes qui decrivent le meme comportement robots.
		\subsection{comportement}
			Un comportement est le terme général pour désigner tout ce que fait le robot : demrrer un moteur, avancer pendant 3s,suivre une ligne, suivre un chemin dans un labyrinthe ...
		\subsection{algo}
			Un algorithme permet de specifier un comportement dans un language proche d'un language naturel
		\subsection{programme}
			Un programme permet de traduire la specification dans le language de programmation du robot

		3 types de comportements:
		\begin{itemize}
			\item elementaire : ie "demarrer moteur"
			\item simple ie "avancer pendant 1 sc"
			\item complexe ie "suivre un chemin dans un labyrinthe"
		\end{itemize}

		Toute solutions non triviale à un problème est d'abord exprimée sous la forme d'un algo, dit de premier niveau, specifiant un algo complexe
	\section{methode}
		Enoncé du problème -> Algo -> programe
		\subsection{etape 1}
			Description de la solution initiale, comportement complexe
		\subsection{etape 2}
			Décomposition en comportement encore plus simple
		\subsection{etape 3}
			Recommencer étape 2 jusqu'à obtenir des comportements élémentaires.
\end{document}